\documentclass[letter]{article}
\usepackage{microtype}
\usepackage{fullpage}
\usepackage{url}

\title{\\
  Computer Science Department\\
  Ball State University\\
  Technical Report 2017--01}

\author{
  Chas Busenburg
  \and
  Paul Gestwicki
  \and
  Ying Liu
  \and
  Jacob Rendall}

\begin{document}

\begin{centering}
{\Large A requirements and design analysis for an interactive online donation interface}\\
\vspace{0.5cm}
Chas Busenburg, Paul Gestwicki\footnote{Corresponding author}, Ying Liu, and Jacob Rendall\\
Computer Science Department\\
\texttt{cwbusenburg@bsu.edu}, 
\texttt{pvgestwicki@bsu.edu},
\texttt{yliu12@bsu.edu},
\texttt{jbrendall@bsu.edu}\\
\vspace{0.5cm}
\today\\
\vspace{0.5cm}
Computer Science Department\\
Ball State University\\
Technical Report 2017--01\\
\end{centering}

\section*{Introduction}

ecoREHAB is a nonprofit organization in Muncie, Indiana, that specializes
in environmentally-friendly, economically-sound rehabilitation of houses.
The organization started as an immersive learning project at Ball
State University in 2009, as a collaboration between the 
City of Muncie Department of Community Development and the 
Ball State College of Architecture and Planning.
ecoREHAB was successful enough to become an independent non-profit, 
although it maintains partnerships with the university.
Their mission statement is given below.

\begin{quote}
  Our Mission is to provide leadership in ecologically sound and
  sustainable rehabilitation of existing housing and
  neighborhoods. ecoREHAB of Muncie, Inc. will engage in activities
  that include: 1)~acquisition and ecologically sustainable
  rehabilitation of affordable housing; 2)~provide resources to aid
  homeowners and in rehabilitation of existing housing following
  sustainable design, material and system strategies; and 3)~other
  activities relative to housing that help the community to achieve
  the triple-bottom-line of economic prosperity, environmental
  protection, and social equity.
\end{quote}

In Spring 2017, Kate Elliott led an undergraduate team of journalism,
marketing, and public relations students in a new immersive learning
partnership with ecoREHAB, with the goal of revising the organization's
branding and Web presence.
ecoREHAB has been primariliy grant-funded, and one of the 
goals for the revised Web site was to accept donations from
the community, including local residents and university alumni.

As a part of Dr. Paul Gestwicki's research and design course (CS 691) at Ball State University, a collaborative effort was established with Kate's class. Their main initiative was on the re-branding process for ecoREHAB while the donation system for the organization was the primary focus for our CS 691 course. The aim of this project was to design and develop an interactive donation system for ecoREHAB, with the main learning goal from the interaction being given below.

\begin{quote}
	Houses can be rehabbed in an environmentally-friendly way, although it takes resources (time, money, knowledge).
\end{quote}

This report presents the design process used by our class for the donation system.

\section*{Design}

\section*{Conclusions and Future Work}

\section*{Appendix}

\end{document}
